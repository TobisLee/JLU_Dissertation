% Copyright (c) 2008-2009 solAvethis
% Copyright (c) 2010-2016,2018 Casper Ti. Vector
% Copyright (c) 2020-2021 jiafeng5513
% Public domain.Aa


% 采用了自定义的(包括大小写不同于原文件的)字体文件名,
% 并改动 ctex.cfg 等配置文件的用户请自行加入 nofonts 选项;
% 其它用户不用加入 nofonts 选项,加入之后反而会产生错误。
% 默认为书籍排版即双面打印,章节从偶数页开始
% oneside:单面打印
% openany:章节从任意页码开始
\documentclass[UTF8,openany,oneside]{jlutex}
% 如果的确须要使脚注按页编号的话,可以去掉后面 footmisc 包的注释。
% 注意:在启用此设定的情况下,可能要多编译一次以产生正确的脚注编号。
\usepackage[perpage]{footmisc}

\usepackage{CJK}

\usepackage{booktabs}
\usepackage{algorithm}
\usepackage{algorithmicx}  
\usepackage{appendix}
\usepackage{algpseudocode}
\usepackage{setspace}

\floatname{algorithm}{算法}
\renewcommand{\algorithmicrequire}{\textbf{输入}}
\renewcommand{\algorithmicensure}{\textbf{输出}}
\newcommand*{\rom}[1]{\expandafter (\romannumeral #1)}

% 使用 biblatex 排版参考文献,并规定其格式(详见 biblatex-caspervector 的文档)。
% 这里按照西文文献在前,中文文献在后排序(“sorting = ecnyt”);
% 若须按照中文文献在前,西文文献在后排序,请设置“sorting = cenyt”;
% 若须按照引用顺序排序,请设置“sorting = none”。
% 若须在排序中实现更复杂的需求,请参考 biblatex-caspervector 的文档。
\usepackage[backend = biber, style = caspervector, utf8]{biblatex}
%\usepackage{subfigure}
% 对于 linespread 值的计算过程有兴趣的同学可以参考 jlutex.cls。
\renewcommand*{\bibfont}{\zihao{5}\linespread{1.27}\selectfont}
% 按学校要求设定参考文献列表的段间距。
\setlength{\bibitemsep}{3bp}

% 设定文档的基本信息。
\jlutexinfo{
	cnumber = {TP391}, %根据自己的专业进行修改
	UnitCode = {10183}, %请根据自己学院的模板进行相应修改
	level = {硕士},
	DegreeCategory = {学术硕士},
	securitylevel = {内部2年},
	cthesisname = {硕士学位论文},ethesisname = {Master Thesis},
	ctitle = {吉林大学学位论文LaTeX模板示例}, etitle = {Sample of LaTexAa Template for Jilin University Dissertation},
	cauthor = {}, eauthor = {},
	address = {吉林省长春市朝阳区前进大街2699号(130012)},
	telephone={},
	studentid = {},
	date = {2020年3月},
	school = {计算机科学与技术学院},
	cmajor = {计算机应用技术}, emajor = {Computer application technology},
	direction = {计算机图形学与数字媒体},
	cmentor = {}, ementor = {},
	ckeywords = {吉林大学,LaTeX,学位论文}, ekeywords = {Jilin University, LaTeX, Dissertation}
}

\addbibresource{ref.bib}	% 载入参考文献数据库(注意不要省略“.bib”)。

% 普通用户可删除此段,并相应地删除 chap/*.tex 中的
% “\jlutexffaq % 中文测试文字。”一行。
\usepackage{color}
\emergencystretch=2em %边缘距离容差

%给目录中的大章节名后面加上点号
\usepackage[subfigure]{tocloft} %模板中用了subfigure,不加此选项会产生冲突
\renewcommand{\cftchapleader}{\cftdotfill{\cftdotsep}}

%对号和错号
\usepackage{amssymb}% http://ctan.org/pkg/amssymb
\usepackage{pifont}% http://ctan.org/pkg/pifont
\newcommand{\cmark}{\ding{51}}%
\newcommand{\xmark}{\ding{55}}%
%修改全局的表格字号
\usepackage{etoolbox}
\BeforeBeginEnvironment{tabular}{\zihao{-5}}
%表格背景色
\usepackage{colortbl}
\usepackage{fancyhdr}

\pagestyle{fancy}
\fancypagestyle{toc}{
\fancyhf{}%
%\fancyfoot{thepage}
\renewcommand{\headrulewidth}{0pt}%
}

\begin{document}{}
	% 以下为正文之前的部分,默认不进行章节编号。
	\frontmatter
	\setcounter{page}{0}		% 重置页码计数器,用大写罗马数字排版此部分页码。\\
	\pagestyle{empty}% 此后到下一 \pagestyle 命令之前不排版页眉或页脚。
	\maketitle					% 生成封面。
	\innertitle
	\setlength{\baselineskip}{30pt}
{
% 此处不用 \specialchap,因为学校要求目录不包括其自己及其之前的内容。
\centerline{\songti\Large吉林大学硕士学位论文原创性声明}
\vskip 1.5cm
% 综合学校的书面要求及 Word 模版来看,版权声明页不用加页眉、页脚。
\thispagestyle{empty}

\songti{本人郑重声明:所呈交学位论文,是本人在指导教师的指导下,独立进行研究工作所取得的成果。
	除文中已经注明引用的内容外,本论文不包含任何其他个人或集体已经发表或撰写过的作品成果。
	对本文的研究做出重要贡献的个人和集体,均已在文中以明确方式标明。
	本人完全意识到本声明的法律结果由本人承担。}

\vskip 5.5cm

\songti{\normalsize
	\hspace{7.0cm}学位论文作者签名:
	
	\hspace{8.8cm}\hspace{2.0cm}年\hspace{1.0cm}月\hspace{1.0cm}日
}
}


	% 原创性声明
	%\cleardoublepage
	\begin{mcopyright}
	\centerline{\heiti\fontsize{20}{20}《中国优秀博硕士学位论文全文数据库》投稿声明}
	\thispagestyle{empty}
	{
		\vskip 1.5cm
		\noindent{研究生院:}\\
		\indent本人同意《中国优秀博硕士学位论文全文数据库》出版章程的内容,
		愿意将本人的学位论文委托研究生院向中国学术期刊(光盘版)电子杂志社的
		《中国优秀博硕士学位论文全文数据库》投稿,
		希望《中国优秀博硕士学位论文全文数据库》给予出版,
		并同意在《中国博硕士学位论文评价数据库》和CNKI系列数据库中使用,
		同意按章程规定享受相关权益。\\ 
	}
\end{mcopyright}	% 投稿声明
	%\cleardoublepage
	
	\pagenumbering{Roman}
	\setcounter{page}{1}		% 重置页码计数器,用大写罗马数字排版此部分页码。\\
	\pagestyle{myheading}	% 此后到下一 \pagestyle 命令之前正常排版页眉和页脚。
	\ctexset{chapter={pagestyle=myheading}}
	\begin{cabstract}
我们致力于简化各位同学在撰写毕业论文以及其他文档时的工作量。
传统毕业论文的撰写流程使用MS Office Word进行编辑。
但由于Word的排版过程较为繁琐,并且在不同的环境中(电脑、系统、Word版本)会出现排版错乱的问题。

因此,我们提出了吉林大学毕业论文LaTex模板以解决上述问题。
运用LaTex可生成高度格式化文档的能力,精准控制毕业论文的排版格式以符合学院审核标准。

测试结果表明,本模板生成的毕业论文可以满足吉林大学针对于毕业论文的格式要求并顺利通过计算机学院的毕业论文格式审核。
此外,针对于基础版本,我们还添加了对于2021年格式要求对模板进行了更新,并使用版式设计技巧对模板进行美化。

此模板仍处于测试阶段,可能会由于Tex Live版本导致奇妙的错误,我们会持续更新模板以满足各位同学的需求。
\end{cabstract}

\begin{eabstract}
% Tranlated by Youdao
We are committed to making it easier for students to write their dissertations and other documents.
The traditional writing process of graduation thesis is edited by MS Office Word.
However, due to the complicated typesetting process of Word, and in different environments (computer, system, Word version), there will be typesetting disorders.

Therefore, we propose a LaTeX template for graduation theses of Jilin University to solve the above problems.
LaTeX provides the ability to generate highly formatted documents, and to precisely control the typesetting of graduation papers to meet the standards of academic review.

The test results show that the graduation thesis generated by this template can meet the format requirements of graduation thesis of Jilin University and pass the examination of graduation thesis format of College of Computer Science.
In addition, for the base version, we also added an update to the template for the format requirements of 2021, and used layout design techniques to beautify the template.
This template is still in beta and may cause some fantastic errors due to the Tex Live version. We will continue to update the template to meet your needs.
\end{eabstract}		% 中西文摘要
	\tableofcontents
	\thispagestyle{myheading}
	\ctexset{chapter={pagestyle=plain}}

	\pagestyle{plain}	% 此后到下一 \pagestyle 命令之前正常排版页眉和页脚。
	\mainmatter% 以下为正文部分,默认要进行章节编号。

	% Copyright (c) 2014,2016,2018 Casper Ti. Vector
% Public domain.

\chapter{引言}
\pkuthssffaq % 中文测试文字。

% vim:ts=4:sw=4
  % 引言
	\chapter{相关工作}

\section{引用格式}
 1.书或专著:
 
	[序号] 作者.书名 [M].版本(第1版不标注).出版地:出版者,出版年.引文所在的起始或起止页码. 
	
 2.期刊(连续出版物)  
 
   [序号] 著者.题(篇)名[J].刊名,出版年,卷号(期号):引文所在的起始或起止页码.
   
 3.会议录、论文集、论文汇编中的析出文献 : 
 
  [序号]析出文献著者.题(篇)名[A].见(英文用In):原文献著者.论文集名[C].出版地:出版者,出版年.引文所在起始或起止页码.
  
  [1]张玉心.重载货车高摩擦系数合成闸瓦的研制和应用[A].见:中国铁道学会编译.国际重载运输协会制动专题讨论会论文集[C].北京:中国铁道学会,1988.242. 
  
  [2]Hunninghaks G W,Gadek J B,Szapiel S V ,et al.The human alveolar macrophage[A].In:Harris C C ed.Cultured human cells and issues in biomedical research[C].New York:Academic Press,1980.54-56.
  
 4.学位论文:
 
	[序号]著者.题(篇)名[D].保存地点:保存单位,年份.引文所在起始或起止页码.
	
 5.专利文献 
 
  [序号]专利所有者.题名[P].专利国别:专利号,出版日期. 
  
 6.技术标准:
 
  [序号]标准编号(标准顺序号-发布年),标准名称[S]. 
  
 7.报纸 
 
  [序号]主要责任者.文献题名[N].报纸名,年-月-日(版次). 
  
 8.科学技术报告
 
  [序号]著者.报告题名[R].出版地:出版者,出版年.页码. 
  
 9.电子文献 
 
  [序号]主要责任者.电子文献题名[电子文献及载体类型标识].电子文献的出处或可获得地址,发表或更新日期/引用日期(任选). 
  
 	电子文献及载体类型标识:数据库(DB),程序(CP),电子公告(EB),磁带(MT),磁盘(DK),光盘(CD),联机网络(OL)
 	
 10.其他未定义类型的文献 
 
  [序号]主要责任者.文献题名[Z].出版地:出版者,出版年.
  

基本字段:除非特别指出,此部分字段在所有类型条目中均可用。 
注:在析出文献条目中,author、editor、translator 专指析出文献的作者、编者、译者。
在 @patent 类条目中,author 也可指专利的持有者。 

\begin{itemize}
	\item bookauthor、booktitle:析出文献所出自文献的作者和题名。 
	\item title:文献题名。 
	\item type:文献类型和电子文献载体标志代码。 
	\item location:出版地,或(在 @patent 类条目中)专利申请地。 
	\item publisher:出版者,或学位论文作者申请学位的单位。 
	\item journal/journaltitle:连续出版物题名,这两个字段是等价的。 
	\item year/date:出版年、日期,这两个字段只需填写一个即可。 
	\item volume:期刊中文献所处的卷号。 
	\item number:期刊中文献所处的期号,或专利的申请号。 
	\item pages:文献页码。 
	\item url:文献的 URL。 
	\item urldate:检索日期,或 URL 的访问日期。 
	\item addendum:补充说明,排版在文献列表中相应条目的最后。
\end{itemize}

\section{添加引用}
请在论文正文中使用supercite命令以引用文献\supercite{Kajiya86}。
对于同时引用多篇的情况\supercite{Houdini, Han07GeoFilter},请在一个supercite命令中用逗号分隔多条文献\supercite{Hu07, Kim2019, Ernst05}。  % 相关工作 
	% Copyright (c) 2014,2016,2018 Casper Ti. Vector
% Public domain.

\chapter{结论和展望}
\pkuthssffaq % 中文测试文字。

% vim:ts=4:sw=4
  % 本文方法
	\chapter{实验和结果}
\section{插入表格}
大多数论文需要使用表格来记录实验数据,根据学校要求,如表~\ref{tab:RTCon}所示,需要使用三线式表格以展示表格内容。
\begin{table}[htb]{
		\centering
		\caption{文档处理实验室主要人员成绩单}
		\label{tab:RTCon}
		\setlength{\tabcolsep}{8mm}
		\begin{tabular}{lcccc}
			\toprule
			姓名 & 语文 & 数学 & 外语 & 政治 \\ 
			\midrule
			安娜 & 100 & 100 & 100 & 100 \\ 
			小羽  & 99 & 95 & 90 & 95 \\ 
			攀攀  & 98  &98 & 94 &99 \\ 
			阿福  & 97& 90 & 90 & 97 \\ 
			大姐头 & 97 &99 & 100 &100 \\ 
			小付 & 97 & 99& 100 & 98\\  
			\midrule
			\textbf{平均} &\textbf{100} & \textbf{100}  & \textbf{100}  & \textbf{100}  \\ 
			\bottomrule
		\end{tabular}\\
	}
\end{table}  % 实验
	\chapter{交叉引用和参考文献}
  % 结论与展望
	% 正文中的附录部分。	
	% 排版参考文献列表。bibintoc 选项使“参考文献”出现在目录中;
	% 如果同时要使参考文献列表参与章节编号,可将“bibintoc”改为“bibnumbered”。
	\normalem
	\printbibliography[heading = bibintoc]
	
	% 各附录。
	\appendix
	\chapter{附件}
附录正文

% vim:ts=4:sw=4


	% 以下为正文之后的部分,默认不进行章节编号。
	% heading=bibintoc    : 将参考文献加入目录中
	% heading=bibnumbered : 参考文献列表参与章节
	 % 作者简介以及科研成果
	\backmatter					                    
	\chapter{作者简介及科研成果}
\section*{作者简介}
此处留白,不需要填写任何内容。

\section*{科研成果}
\vskip 10pt
\begin{itemize}[itemsep= 7pt, labelsep= 10pt, leftmargin = 25pt, topsep = 10pt]
	\small
	\setlength{\baselineskip}{16pt}
	\item[{$\left[1\right]$}]
	除导师外第一作者. \textit{"Paper name"} [C]. In: \textit{Conference name}, \textbf{Year}, pages-pages.  doi: \texttt{doi}. (EI Accession number: \texttt{accession number}).
	
	\item[{$\left[2\right]$}]
	除导师外第一作者. "\textit{Paper name}" [J]. \textit{Journal name}, \textbf{Year}. doi: \texttt{doi}. (CCF C类期刊).
	
	\item[{$\left[3\right]$}]
	基于***的***系统V1.0. 中国. 计算机软件著作权. 登记号:*********.
\end{itemize}
	%\chapter{致谢}
感谢安娜学姐制作Latex模板。
感谢P\_Lee在对于修改提出的各种建议以及帮助。
感谢小付提出的论文撰写建议。
感谢Ris对于论文模板内容的更新。

同时,感谢各位参与测试的同学。也希望各位同学可以积极参与到模板的测试和修改中来。
% vim:ts=4:sw=4
	% 致谢

\end{document}

% vim:ts=4:sw=4
