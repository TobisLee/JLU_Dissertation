% Copyright (c) 2008-2009 solvethis
% Copyright (c) 2010-2016,2018 Casper Ti. Vector
% Copyright (c) 2020-2021 jiafeng5513
% Public domain.
%
% 使用前请先仔细阅读 jlutex 和 biblatex-caspervector 的文档,
% 特别是其中的 FAQ 部分和用红色强调的部分。
% 两者可在终端/命令提示符中用
%   texdoc jlutex
%   texdoc biblatex-caspervector
% 调出。

% 采用了自定义的(包括大小写不同于原文件的)字体文件名,
% 并改动 ctex.cfg 等配置文件的用户请自行加入 nofonts 选项;
% 其它用户不用加入 nofonts 选项,加入之后反而会产生错误。
% 默认为书籍排版即双面打印,章节从偶数页开始
% oneside:单面打印
% openany:章节从任意页码开始
\documentclass[UTF8,openany,oneside]{jlutex}
% 如果的确须要使脚注按页编号的话,可以去掉后面 footmisc 包的注释。
% 注意:在启用此设定的情况下,可能要多编译一次以产生正确的脚注编号。
%\usepackage[perpage]{footmisc}

% 使用 biblatex 排版参考文献,并规定其格式(详见 biblatex-caspervector 的文档)。
% 这里按照西文文献在前,中文文献在后排序(“sorting = ecnyt”);
% 若须按照中文文献在前,西文文献在后排序,请设置“sorting = cenyt”;
% 若须按照引用顺序排序,请设置“sorting = none”。
% 若须在排序中实现更复杂的需求,请参考 biblatex-caspervector 的文档。
\usepackage[backend = biber, style = caspervector, utf8, sorting = none]{biblatex}

% 对于 linespread 值的计算过程有兴趣的同学可以参考 jlutex.cls。
\renewcommand*{\bibfont}{\zihao{5}\linespread{1.27}\selectfont}
% 按学校要求设定参考文献列表的段间距。
\setlength{\bibitemsep}{3bp}

% 设定文档的基本信息。
\jlutexinfo{
	cnumber = {TP391},
	UnitCode = {10183},
	level = {硕士},
	DegreeCategory = {学术硕士},
	securitylevel = {公开},
	cthesisname = {硕士学位论文},ethesisname = {Master Thesis},
	ctitle = {中文标题}, etitle = {English Title},
	cauthor = {贝多芬}, eauthor = {Beethoven},
	address = {贝克大街221B},
	telephone={13800000000},
	studentid = {0123456789},
	date = {2020年3月},
	school = {计算机科学与技术学院},
	cmajor = {计算机应用技术}, emajor = {Computer application technology},
	direction = {计算机图形学与数字媒体},
	cmentor = {傅里叶教授}, ementor = {Prof.\ Fourier},
	ckeywords = {关键词1,关键词2}, ekeywords = {First, Second}
}

\addbibresource{ref.bib}	% 载入参考文献数据库(注意不要省略“.bib”)。

% 普通用户可删除此段,并相应地删除 chap/*.tex 中的
% “\jlutexffaq % 中文测试文字。”一行。
\usepackage{color}
\def\jlutexffaq{%
	\emph{\textcolor{red}{jlutex 文档模版最常见问题:}}

	\texttt{\string\cite}、\texttt{\string\parencite} %
	和 \texttt{\string\supercite} 三个命令分别产生%
	未格式化的、带方括号的和上标且带方括号的引用标记:%
	\cite{test-en},\parencite{test-zh}、\supercite{test-en, test-zh}。

	若要避免章末空白页,请在调用 jlutex 文档类时加入 \texttt{openany} 选项。

	如果编译时不出参考文献,
	请参考 \texttt{texdoc jlutex}“问题及其解决”一章
	“上游宏包可能引起的问题”一节中关于 biber 的说明。%
}

\begin{document}{}
	% 以下为正文之前的部分,默认不进行章节编号。
	\frontmatter
	
	\pagestyle{empty}% 此后到下一 \pagestyle 命令之前不排版页眉或页脚。

	\maketitle					% 生成封面。
	\setlength{\baselineskip}{30pt}
{
% 此处不用 \specialchap,因为学校要求目录不包括其自己及其之前的内容。
\centerline{\songti\Large吉林大学硕士学位论文原创性声明}
\vskip 1.5cm
% 综合学校的书面要求及 Word 模版来看,版权声明页不用加页眉、页脚。
\thispagestyle{empty}

\songti{本人郑重声明:所呈交学位论文,是本人在指导教师的指导下,独立进行研究工作所取得的成果。
	除文中已经注明引用的内容外,本论文不包含任何其他个人或集体已经发表或撰写过的作品成果。
	对本文的研究做出重要贡献的个人和集体,均已在文中以明确方式标明。
	本人完全意识到本声明的法律结果由本人承担。}

\vskip 5.5cm

\songti{\normalsize
	\hspace{7.0cm}学位论文作者签名:
	
	\hspace{8.8cm}\hspace{2.0cm}年\hspace{1.0cm}月\hspace{1.0cm}日
}
}


	% 原创性声明
	%\cleardoublepage
	\begin{mcopyright}
	\centerline{\heiti\fontsize{20}{20}《中国优秀博硕士学位论文全文数据库》投稿声明}
	\thispagestyle{empty}
	{
		\vskip 1.5cm
		\noindent{研究生院:}\\
		\indent本人同意《中国优秀博硕士学位论文全文数据库》出版章程的内容,
		愿意将本人的学位论文委托研究生院向中国学术期刊(光盘版)电子杂志社的
		《中国优秀博硕士学位论文全文数据库》投稿,
		希望《中国优秀博硕士学位论文全文数据库》给予出版,
		并同意在《中国博硕士学位论文评价数据库》和CNKI系列数据库中使用,
		同意按章程规定享受相关权益。\\ 
	}
\end{mcopyright}	% 投稿声明
	%\cleardoublepage
	\pagestyle{plain}	% 此后到下一 \pagestyle 命令之前正常排版页眉和页脚。
	\setcounter{page}{0}		% 重置页码计数器,用大写罗马数字排版此部分页码。
	\pagenumbering{Roman}
	
	\begin{cabstract}
	中文摘要正文
	\pkuthssffaq
\end{cabstract}
\begin{eabstract}
	Test of the English abstract.
\end{eabstract}			% 中西文摘要
	\tableofcontents			% 自动生成目录。
	\mainmatter% 以下为正文部分,默认要进行章节编号。
	% Copyright (c) 2014,2016,2018 Casper Ti. Vector
% Public domain.

\chapter{引言}
\pkuthssffaq % 中文测试文字。

% vim:ts=4:sw=4
% 各章节。
	\chapter{相关工作}

\section{引用格式}
 1.书或专著:
 
	[序号] 作者.书名 [M].版本(第1版不标注).出版地:出版者,出版年.引文所在的起始或起止页码. 
	
 2.期刊(连续出版物)  
 
   [序号] 著者.题(篇)名[J].刊名,出版年,卷号(期号):引文所在的起始或起止页码.
   
 3.会议录、论文集、论文汇编中的析出文献 : 
 
  [序号]析出文献著者.题(篇)名[A].见(英文用In):原文献著者.论文集名[C].出版地:出版者,出版年.引文所在起始或起止页码.
  
  [1]张玉心.重载货车高摩擦系数合成闸瓦的研制和应用[A].见:中国铁道学会编译.国际重载运输协会制动专题讨论会论文集[C].北京:中国铁道学会,1988.242. 
  
  [2]Hunninghaks G W,Gadek J B,Szapiel S V ,et al.The human alveolar macrophage[A].In:Harris C C ed.Cultured human cells and issues in biomedical research[C].New York:Academic Press,1980.54-56.
  
 4.学位论文:
 
	[序号]著者.题(篇)名[D].保存地点:保存单位,年份.引文所在起始或起止页码.
	
 5.专利文献 
 
  [序号]专利所有者.题名[P].专利国别:专利号,出版日期. 
  
 6.技术标准:
 
  [序号]标准编号(标准顺序号-发布年),标准名称[S]. 
  
 7.报纸 
 
  [序号]主要责任者.文献题名[N].报纸名,年-月-日(版次). 
  
 8.科学技术报告
 
  [序号]著者.报告题名[R].出版地:出版者,出版年.页码. 
  
 9.电子文献 
 
  [序号]主要责任者.电子文献题名[电子文献及载体类型标识].电子文献的出处或可获得地址,发表或更新日期/引用日期(任选). 
  
 	电子文献及载体类型标识:数据库(DB),程序(CP),电子公告(EB),磁带(MT),磁盘(DK),光盘(CD),联机网络(OL)
 	
 10.其他未定义类型的文献 
 
  [序号]主要责任者.文献题名[Z].出版地:出版者,出版年.
  

基本字段:除非特别指出,此部分字段在所有类型条目中均可用。 
注:在析出文献条目中,author、editor、translator 专指析出文献的作者、编者、译者。
在 @patent 类条目中,author 也可指专利的持有者。 

\begin{itemize}
	\item bookauthor、booktitle:析出文献所出自文献的作者和题名。 
	\item title:文献题名。 
	\item type:文献类型和电子文献载体标志代码。 
	\item location:出版地,或(在 @patent 类条目中)专利申请地。 
	\item publisher:出版者,或学位论文作者申请学位的单位。 
	\item journal/journaltitle:连续出版物题名,这两个字段是等价的。 
	\item year/date:出版年、日期,这两个字段只需填写一个即可。 
	\item volume:期刊中文献所处的卷号。 
	\item number:期刊中文献所处的期号,或专利的申请号。 
	\item pages:文献页码。 
	\item url:文献的 URL。 
	\item urldate:检索日期,或 URL 的访问日期。 
	\item addendum:补充说明,排版在文献列表中相应条目的最后。
\end{itemize}

\section{添加引用}
请在论文正文中使用supercite命令以引用文献\supercite{Kajiya86}。
对于同时引用多篇的情况\supercite{Houdini, Han07GeoFilter},请在一个supercite命令中用逗号分隔多条文献\supercite{Hu07, Kim2019, Ernst05}。
	% Copyright (c) 2014,2016,2018 Casper Ti. Vector
% Public domain.

\chapter{结论和展望}
\pkuthssffaq % 中文测试文字。

% vim:ts=4:sw=4


	% 正文中的附录部分。
	\appendix
	% 排版参考文献列表。bibintoc 选项使“参考文献”出现在目录中;
	% 如果同时要使参考文献列表参与章节编号,可将“bibintoc”改为“bibnumbered”。
	\printbibliography[heading = bibintoc]
	% 各附录。
	% Copyright (c) 2014,2016 Casper Ti. Vector
% Public domain.

\chapter{附件}
\pkuthssffaq % 中文测试文字。

% vim:ts=4:sw=4


	% 以下为正文之后的部分,默认不进行章节编号。
	\backmatter
	% heading=bibintoc    : 将参考文献加入目录中
	% heading=bibnumbered : 参考文献列表参与章节编号
	
	\printbibliography[title = {参考文献}, heading = bibnumbered]     % 参考文献
						% 作者简介以及科研成果
	% Copyright (c) 2014,2016 Casper Ti. Vector
% Public domain.

\chapter{致谢}
\pkuthssffaq % 中文测试文字。

% vim:ts=4:sw=4
	% 致谢

\end{document}

% vim:ts=4:sw=4
