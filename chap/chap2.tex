\chapter{相关工作}

\section{引用格式}
 1.书或专著:
 
	[序号] 作者.书名 [M].版本(第1版不标注).出版地:出版者,出版年.引文所在的起始或起止页码. 
	
 2.期刊(连续出版物)  
 
   [序号] 著者.题(篇)名[J].刊名,出版年,卷号(期号):引文所在的起始或起止页码.
   
 3.会议录、论文集、论文汇编中的析出文献 : 
 
  [序号]析出文献著者.题(篇)名[A].见(英文用In):原文献著者.论文集名[C].出版地:出版者,出版年.引文所在起始或起止页码.
  
  [1]张玉心.重载货车高摩擦系数合成闸瓦的研制和应用[A].见:中国铁道学会编译.国际重载运输协会制动专题讨论会论文集[C].北京:中国铁道学会,1988.242. 
  
  [2]Hunninghaks G W,Gadek J B,Szapiel S V ,et al.The human alveolar macrophage[A].In:Harris C C ed.Cultured human cells and issues in biomedical research[C].New York:Academic Press,1980.54-56.
  
 4.学位论文:
 
	[序号]著者.题(篇)名[D].保存地点:保存单位,年份.引文所在起始或起止页码.
	
 5.专利文献 
 
  [序号]专利所有者.题名[P].专利国别:专利号,出版日期. 
  
 6.技术标准:
 
  [序号]标准编号(标准顺序号-发布年),标准名称[S]. 
  
 7.报纸 
 
  [序号]主要责任者.文献题名[N].报纸名,年-月-日(版次). 
  
 8.科学技术报告
 
  [序号]著者.报告题名[R].出版地:出版者,出版年.页码. 
  
 9.电子文献 
 
  [序号]主要责任者.电子文献题名[电子文献及载体类型标识].电子文献的出处或可获得地址,发表或更新日期/引用日期(任选). 
  
 	电子文献及载体类型标识:数据库(DB),程序(CP),电子公告(EB),磁带(MT),磁盘(DK),光盘(CD),联机网络(OL)
 	
 10.其他未定义类型的文献 
 
  [序号]主要责任者.文献题名[Z].出版地:出版者,出版年.
  

基本字段:除非特别指出,此部分字段在所有类型条目中均可用。 
注:在析出文献条目中,author、editor、translator 专指析出文献的作者、编者、译者。
在 @patent 类条目中,author 也可指专利的持有者。 

\begin{itemize}
	\item bookauthor、booktitle:析出文献所出自文献的作者和题名。 
	\item title:文献题名。 
	\item type:文献类型和电子文献载体标志代码。 
	\item location:出版地,或(在 @patent 类条目中)专利申请地。 
	\item publisher:出版者,或学位论文作者申请学位的单位。 
	\item journal/journaltitle:连续出版物题名,这两个字段是等价的。 
	\item year/date:出版年、日期,这两个字段只需填写一个即可。 
	\item volume:期刊中文献所处的卷号。 
	\item number:期刊中文献所处的期号,或专利的申请号。 
	\item pages:文献页码。 
	\item url:文献的 URL。 
	\item urldate:检索日期,或 URL 的访问日期。 
	\item addendum:补充说明,排版在文献列表中相应条目的最后。
\end{itemize}

\section{添加引用}
请在论文正文中使用supercite命令以引用文献\supercite{Kajiya86}。
对于同时引用多篇的情况\supercite{Houdini, Han07GeoFilter},请在一个supercite命令中用逗号分隔多条文献\supercite{Hu07, Kim2019, Ernst05}。